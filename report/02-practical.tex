\chapter{Практическая часть}

\section{Используемые инструменты}

Для реализации нечеткой системы использована библиотека \texttt{scikit-fuzzy} и язык Python 3.  
Основные зависимости:
\begin{itemize}
    \item \texttt{numpy} — численные вычисления;
    \item \texttt{matplotlib} — визуализация;
    \item \texttt{scikit-fuzzy} — функции принадлежности и дефаззификация.
\end{itemize}

\section{Результаты}

На рисунке~\ref{img:result} показана динамика движения автомобилей.  
Автопилот плавно регулирует скорость, удерживая требуемую дистанцию от лидера даже при изменении его скорости.

% \includeimage
%     {result}
%     {f}
%     {h}
%     {0.75\textwidth}
%     {Результаты моделирования движения автомобилей}

Среднеквадратичная ошибка поддержания дистанции составила:
\[
MSE = 0.153.
\]

Время моделирования при шаге интегрирования \( dt = 0.1 \) и \( T = 100 \, c \) составило менее 1 секунды.

\section{Особенности реализации}

\begin{itemize}
    \item Импликация Ларсена реализована вручную как масштабирование функции принадлежности по вертикали.
    \item Используется произведение степеней истинности для расчета силы активации (\( \alpha = \mu_1 \times \mu_2 \)).
    \item Дефаззификация выполняется методом центра тяжести (\texttt{fuzz.defuzz(..., 'centroid')}).
    \item Система легко расширяется на \( X=3 \) при добавлении новых входных переменных (например, бокового смещения).
\end{itemize}

\chapter{Теоретическая часть}

\section{Постановка задачи}

В одномерном пространстве (\( X=1 \)) рассматриваются два автомобиля: лидер, управляемый пользователем, и автомобиль-автопилот.  
Автопилот должен следовать за лидером, поддерживая постоянную дистанцию \( D_{жел} \), не имея информации о скорости лидера.  
Известно только текущее расстояние между автомобилями.  
Требуется определить необходимую скорость автопилота \( v_{auto} \) на основе нечеткого логического вывода.

Определение ускорения запрещено.  
Входными переменными являются:
\begin{itemize}
    \item ошибка по расстоянию \( e = D_{тек} - D_{жел} \);
    \item изменение ошибки \( \Delta e = \frac{de}{dt} \).
\end{itemize}

Выходная переменная — скорость автопилота \( v_{auto} \).

\section{Этапы нечеткого логического вывода}

Нечеткий логический вывод состоит из следующих этапов:

\begin{enumerate}
    \item \textbf{Фаззификация} — преобразование четких входных значений \( e \) и \( \Delta e \) в степени принадлежности нечетким подмножествам.
    \item \textbf{Применение базы правил} — вычисление степени активации каждого правила на основе входных значений.
    \item \textbf{Импликация} — формирование выходных нечетких множеств в соответствии с вычисленной степенью активации \(\alpha\).
    \item \textbf{Агрегация} — объединение всех выходных множеств.
    \item \textbf{Дефаззификация} — преобразование агрегированного нечеткого множества в четкое значение \( v_{auto} \).
\end{enumerate}

\section{Функции принадлежности}

Для входных и выходных переменных были выбраны следующие функции принадлежности:

\begin{itemize}
    \item Ошибка расстояния \texttt{Negative}: \texttt{Zero}, \texttt{Positive};
    \item Изменение ошибки \texttt{Negative}: \texttt{Zero}, \texttt{Positive};
    \item Скорость \texttt{Slow}: \texttt{Medium}, \texttt{Fast}.
\end{itemize}

Для задания функций использовались треугольные формы:
\begin{align*}
\texttt{trimf}(x; a,b,c) &= 
\begin{cases}
0, & x < a,\\
\frac{x-a}{b-a}, & a \le x < b,\\
\frac{c-x}{c-b}, & b \le x < c,\\
0, & x \ge c.
\end{cases}
\end{align*}

% \includeimage
%     {membership_example}
%     {f}
%     {h}
%     {0.75\textwidth}
%     {Пример функций принадлежности для переменных error, delta и speed}

\section{База правил}

Правила нечеткого вывода описывают зависимость между ошибками и требуемой скоростью:

\begin{enumerate}
    \item ЕСЛИ distance\_error = Positive И delta\_distance = Negative ТО v\_follower = Slow;
    \item ЕСЛИ distance\_error = Positive И delta\_distance = Zero ТО v\_follower = Slow;
    \item ЕСЛИ distance\_error = Positive И delta\_distance = Positive ТО v\_follower = Medium;
    \item ЕСЛИ distance\_error = Zero И delta\_distance = Negative ТО v\_follower = Medium;
    \item ЕСЛИ distance\_error = Zero И delta\_distance = Zero ТО v\_follower = Medium;
    \item ЕСЛИ distance\_error = Zero И delta\_distance = Positive ТО v\_follower = Fast;
    \item ЕСЛИ distance\_error = Negative И delta\_distance = Negative ТО v\_follower = Medium;
    \item ЕСЛИ distance\_error = Negative И delta\_distance = Zero ТО v\_follower = Fast;
    \item ЕСЛИ distance\_error = Negative И delta\_distance = Positive ТО v\_follower = Fast.
\end{enumerate}

\section{Алгоритм логического вывода}

Для варианта лабораторной работы используется алгоритм Ларсена:
\begin{equation*}
    \mu_{A \wedge B}(z) = \mu_A(x) \cdot \mu_B(y),
\end{equation*}

где:\\
$\mu_A(x)$ — функция принадлежности входной переменной $x$ к множеству $A$;\\
$\mu_B(y)$ — функция принадлежности входной переменной $y$ к множеству $B$.

\section{Алгоритм дефаззификации}

Для получения четкого значения скорости используется метод центра тяжести (Centroid method):
\[
v = \frac{\int z \cdot \mu(z) \, dz}{\int \mu(z) \, dz}.
\]

В качестве альтернативного метода можно применять метод среднего максимума (Mean of maxima, MOM):
\[
v = \frac{z_{\min} + z_{\max}}{2}, \quad z_{\min,\max} \in \{z | \mu(z) = \max(\mu)\}.
\]
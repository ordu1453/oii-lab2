\chapter{Практическая часть}

\section{Используемые инструменты}

Для реализации нечеткой системы использована библиотека \texttt{scikit-fuzzy} и язык
 Python 3.  
Основные зависимости:
\begin{itemize}
    \item \texttt{numpy} — численные вычисления;
    \item \texttt{matplotlib} — визуализация;
    \item \texttt{scikit-fuzzy} — функции принадлежности и дефаззификация.
\end{itemize}

\section{Функции принадлежности}

На рисунках \ref{img:de}-\ref{img:vf} представлены функции 
принадлежности нечетких переменных.

    \includeimage
        {de}
        {f}
        {h}
        {0.55\textwidth}
        {Функция принадлежности расстояния между автомобилями}

    \includeimage
        {dd}
        {f}
        {h}
        {0.55\textwidth}
        {Функция принадлежности изменения расстояния}

    \includeimage
        {vf}
        {f}
        {h}
        {0.55\textwidth}
        {Функция принадлежности скорости автопилота}

        \newpage

\section{Реализация алгоритма Ларсена}

В библиотеке scikit-fuzzy в качестве алгоритма логического вывода применяется
алгоритм Мамдани, без возможности выбора альтернативы. 
Одним из вариантов реализации алгоритма Ларсена с использованием
scikit-fuzzy, требует модификации исходного кода библиотеки.

\includelistingpretty
{rule.py} % Имя файла с расширением (файл должен быть расположен в директории inc/lst/)
{python} % Язык программирования (необязательный аргумент)
{Исходный код библиотеки scikit-fuzzy, rule.py} % Подпись листинга

Как видно из листинга \ref{lst:rule.py}, выходом функции логического И является 
минимальное из двух значений \lstinline|and_func = np.fmin| (по Мамдани).
Заменив этот кусок кода на \lstinline|and_func=np.multiply|, получим алгоритм вывода
Ларсена.

\section{Результаты}

\subsection{Моделирование реакции системы на единичное ступенчатое
 воздействие}

На рисунках \ref{img:disErr_com}-\ref{img:izmCor_com} показаны результаты
 моделирования системы при единичном ступенчатом воздействии.

    \includeimage
        {disErr_com}
        {f}
        {h}
        {0.65\textwidth}
        {График изменения ошибки системы}

    \includeimage
        {izmScor_com}
        {f}
        {h}
        {0.65\textwidth}
        {График изменения скорости автомобилей}

        
    \includeimage
        {izmCor_com}
        {f}
        {h}
        {0.65\textwidth}
        {График изменения координат автомобилей}

\newpage

\subsection{Моделирование реакции системы на двухступенчатое воздействие}

На рисунках \ref{img:disErr_sl}-\ref{img:izmCor_sl} показаны результаты
 моделирования системы при двухступенчатом воздействии.

    \includeimage
        {disErr_sl}
        {f}
        {h}
        {0.65\textwidth}
        {График изменения ошибки системы}

    \includeimage
        {izmScor_sl}
        {f}
        {h}
        {0.65\textwidth}
        {График изменения скорости автомобилей}

    \includeimage
        {izmCor_sl}
        {f}
        {h}
        {0.65\textwidth}
        {График изменения координат автомобилей}

\newpage

Среднеквадратичная ошибка поддержания дистанции при моделировании системы составила:
\[
MSE = 0.0540.
\]

Шаг интегрирования \( dt = 0.1 \) и время моделирования \( T = 50 \, c \).


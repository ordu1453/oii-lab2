\chapter{Теоретическая часть}

\section{Постановка задачи}

В одномерном пространстве (\( X=1 \)) рассматриваются два автомобиля: лидер, управляемый пользователем, и автомобиль-автопилот.  
Автопилот должен следовать за лидером, поддерживая постоянную дистанцию \( D_{жел} \), не имея информации о скорости лидера.  
Известно только текущее расстояние между автомобилями.  
Требуется определить необходимую скорость автопилота \( v_{auto} \) на основе нечеткого логического вывода.

Определение ускорения запрещено.  
Входными переменными являются:
\begin{itemize}
    \item ошибка по расстоянию \( e = D_{тек} - D_{жел} \);
    \item изменение ошибки \( \Delta e = \frac{de}{dt} \).
\end{itemize}

Выходная переменная — скорость автопилота \( v_{auto} \).

\section{Этапы нечеткого логического вывода}

Нечеткий логический вывод состоит из следующих этапов:

\begin{enumerate}
    \item \textbf{Фаззификация} — преобразование четких входных значений \( e \) и \( \Delta e \) в степени принадлежности нечетким подмножествам.
    \item \textbf{Применение базы правил} — вычисление степени активации каждого правила на основе входных значений.
    \item \textbf{Импликация} — формирование выходных нечетких множеств в соответствии с вычисленной степенью активации \(\alpha\).
    \item \textbf{Агрегация} — объединение всех выходных множеств.
    \item \textbf{Дефаззификация} — преобразование агрегированного нечеткого множества в четкое значение \( v_{auto} \).
\end{enumerate}

\section{Функции принадлежности}

Для входных и выходных переменных были выбраны следующие функции принадлежности (рис.~\ref{img:membership}):

\begin{itemize}
    \item Ошибка расстояния \texttt{error}: \texttt{too\_close}, \texttt{normal}, \texttt{far};
    \item Изменение ошибки \texttt{delta}: \texttt{approaching}, \texttt{steady}, \texttt{moving\_away};
    \item Скорость \texttt{speed}: \texttt{slow}, \texttt{medium}, \texttt{fast}.
\end{itemize}

Для задания функций использовались трапециевидные и треугольные формы:
\begin{align*}
\texttt{trapmf}(x; a,b,c,d) &= 
\begin{cases}
0, & x < a,\\
\frac{x-a}{b-a}, & a \le x < b,\\
1, & b \le x \le c,\\
\frac{d-x}{d-c}, & c < x < d,\\
0, & x \ge d,
\end{cases}\\
\texttt{trimf}(x; a,b,c) &= 
\begin{cases}
0, & x < a,\\
\frac{x-a}{b-a}, & a \le x < b,\\
\frac{c-x}{c-b}, & b \le x < c,\\
0, & x \ge c.
\end{cases}
\end{align*}

% \includeimage
%     {membership_example}
%     {f}
%     {h}
%     {0.75\textwidth}
%     {Пример функций принадлежности для переменных error, delta и speed}

\section{База правил}

Правила нечеткого вывода описывают зависимость между ошибками и требуемой скоростью:

\begin{enumerate}
    \item ЕСЛИ error = too\_close И delta = approaching ТО speed = slow;
    \item ЕСЛИ error = too\_close И delta = steady ТО speed = slow;
    \item ЕСЛИ error = too\_close И delta = moving\_away ТО speed = medium;
    \item ЕСЛИ error = normal И delta = approaching ТО speed = slow;
    \item ЕСЛИ error = normal И delta = steady ТО speed = medium;
    \item ЕСЛИ error = normal И delta = moving\_away ТО speed = fast;
    \item ЕСЛИ error = far И delta = approaching ТО speed = medium;
    \item ЕСЛИ error = far И delta = steady ТО speed = fast;
    \item ЕСЛИ error = far И delta = moving\_away ТО speed = fast.
\end{enumerate}

\section{Алгоритм импликации Ларсена}

Для варианта лабораторной работы используется \textbf{импликация Ларсена}:
\[
\mu_{B'}(y) = \alpha \times \mu_B(y),
\]
где \(\alpha\) — степень истинности предпосылки.

В данной работе \(\alpha\) определяется как произведение степеней истинности входных условий:
\[
\alpha = \mu_{A_1}(x_1) \cdot \mu_{A_2}(x_2),
\]
что обеспечивает более плавное изменение выходного множества по сравнению с методом Мамдани, где используется минимум.

\section{Алгоритм дефаззификации}

Для получения четкого значения скорости используется метод \textbf{центра тяжести} (centroid):
\[
v = \frac{\int y \cdot \mu(y) \, dy}{\int \mu(y) \, dy}.
\]

В качестве альтернативного метода можно применять \textbf{метод среднего максимума (mean of maxima)}:
\[
v = \frac{y_{\min} + y_{\max}}{2}, \quad y_{\min,\max} \in \{y | \mu(y) = \max(\mu)\}.
\]